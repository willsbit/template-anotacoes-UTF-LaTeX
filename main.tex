%%% Template para anotações de aula
%%% Feito por Willian, com base no template de  Mikhail Klassen

\documentclass[12pt,a4paper, brazil]{article}

%%%%%%% INFORMAÇÕES DO CABEÇALHO
\newcommand{\workingDate}{\textsc{\selectlanguage{portuguese}\today}}
\newcommand{\userName}{Cód. da matéria}
\newcommand{\institution}{UTFPR-PB}
\usepackage{researchdiary_png}

\begin{document}

\title{Anotações\ da\ aula}
%% {\large \today}\\[5mm] %% se quiser colocar data

\section*{Template para anotações de aula}

\subsection*{Aula 1}


%% para citar no texto, use \textcite{citacao} para ficar no formato Fulano (Ano), ou use \cite{citacao} para citar  no formato (FULANO, Ano)
Segundo \textcite{citacao-exemplo}, \lipsum[66] \cite{citacao-exemplo}

\par
Para equações, listas, etc., basta usar os environments padrão do LaTeX.

\begin{equation}
a^2 = b^2 + c^2
\end{equation}
onde $a$, $b$ e $c$ são lados de um triângulo retângulo........

\subsubsection*{Subseção da aula 1}

\lipsum[66]

\myparagraph{Parágrafo da subseção da aula 1}
\lipsum[66]
\par A \autoref{fig:exemplo} é uma imagem......

\begin{figure}[!htb]
    \centering
    \caption{Figura exemplo}
    \includegraphics[width=0.5\textwidth]{logo_utf.png}
    \caption*{Fonte: Autoria própria.} %% isso é levemente uma gambiarra, mas funciona para o propósito desse template.
    \label{fig:exemplo}
\end{figure}

\mysubparagraph{Subparágrafo da subseção da aula 1}
\lipsum[66]

%%% as referências devem estar em formato bibTeX no arquivo referencias.bib
\printbibliography

\end{document}